\section{Methodology}
	
\subsection{Privacy}	
The definition of privacy which is presented by Zureik et al. (2010) captures most relevant dimensions that usually can be found: \\
\begin{enumerate}
	\item the right to be let alone (no intrusion by third parties
	if not wanted),
	\item limited access to the self (e.g. refuse to open the door of your house if you don’t want to), 
	\item secrecy (e.g. correspondence, such as e-mail must remain confidential)),
	\item control of personal information (e.g. being able to know what the super market or the tax	office have collected about you), \item personhood (e.g. you may experiment with your identity,
	if that makes you happy), and 
	\item intimacy (no one can enter the very private sphere around you if not allowed to do so).
\end{enumerate} 
~\\
These six dimensions are loosely related to a classical distinction made by Westin (1967), differentiating between various spheres of privacy: solitude, intimacy, reserve and anonymity. These two definitions show that privacy is a relatively broad concept that captures various aspects of ordinary life.\\
When looking from the angle of information and communication technology, the focus is primarily on the informational dimension of privacy. It deals with personal data that floats	around and that may be used - outside direct control - of the subject to which the data relate. In this context one can define privacy as the right to control the release of personal information about oneself, even when that data is collected and stored by a third party.\\
Although proper security mechanisms to protect personal information are necessary to prevent unauthorised access and use of that information, it is important to understand that privacy
is not the same as confidentiality (or data security). The right to privacy also stipulates that	personal data is only collected when this is necessary, that no more personal data is collected than needed, and that this data should only be used for the purpose for which it was originally	collected. Also, people have the right to view and update personal information held by others about themselves.
	
\subsection{Legal framework}
One typical approach to protecting privacy is using legal measures. Most prominent is the Universal Declaration of Human Rights. In article 12 it is stated that ‘No one shall be subjected to arbitrary interference with his privacy, family, home or correspondence, nor to attacks upon his honour and reputation. Everyone has the right to the protection of the law against	such interference or attacks.’ This right to protection is repeated in the European Charter of	Fundamental Rights (2000) which states that everyone has the right to respect for his or her private and family life, home and communications (article 7) and everyone has the right to protection of personal data (article 8). The distinction between both articles is telling: while article 7 relates to privacy, article 8 relates to personal data. In our approach we would consider article 8 to relate to the informational dimension of privacy. As we will show, in current	practices both articles become more closely related.
	
It is especially the right to protection of personal data that is of relevance for this research. This right has led to some important European Directives. The two most relevant ones are	the directive ‘on the protection of individuals with regard to the processing of personal data	and on the free movement of such data’ (Directive 95/46/EC) and the directive ‘concerning the protection of personal data and the protection of privacy in the electronic communications sector’ (Directive 2002/58/EC). This latter Directive has meanwhile been inserted in a larger Directive (2009/136/EC) which integrates two directives (one on Universal Service as well) and a Regulation on consumer laws. Some aspects of the 2002/58/EC directive have been elaborated in more detail in the new Directive.\\
	
Directive 95/46/EC formulates a number of criteria for the lawful processing of personal data. These criteria refer in turn to a set of privacy principles which has been formulated by the OECD already in 1980 (OECD 2011). Each processing of personal data means a possible infringement on the privacy of the people whose data are collected and processed. This is only allowed when it can be justified on the following aspects:
\begin{itemize}
	\setlength\itemsep{0em}
	\item it serves a legitimate aim
	\item it is lawful
	\item collecting and processing the data is necessary (for instance to deliver a service or a product).
\end{itemize}
	
When collecting data can be justified, the collection process itself should meet the following criteria:
\begin{enumerate}
	\item it should serve a clear purpose
	\item the purpose cannot be achieved in another, less invasive way (subsidiarity)
	\item the data collection should be proportionate (not excessive and in line with the purpose)
	\item safeguards should be in place (security measures, quality of data)
	\item the rights of the ‘data subjects’ should be guaranteed (informed consent, right to access and correct the data)
\end{enumerate}

\subsection{Logging interactions: UsaProxy}
UsaProxy is based upon an HTTP proxy approach. Logging is automatically done on an intermediate computer lying between the web browser and the web servers while multiple users surf the web. The assumption is that all page requests, the browsers make, go through the proxy.
	
\begin{figure}[h] 
	\centering
	\includegraphics[scale=0.9]{images/proxy.png}
	\caption{UsaProxy}
\end{figure}
	
\begin{itemize}	
	\item The users are fulfilling their tasks visiting web pages while UsaProxy is registered as proxy in the user’s browser properties
	\item When a user’s browser requests UsaProxy to forward him a specific web page, the respective page is fetched from the web server and prepared with special JavaScript code before it is transmitted to the user
	\item During the user’s visit on the page all his actions (such as page requests, mouseclicks, mouse movements, scrolling, keystrokes, etc.) are monitored and continuously transmitted to the UsaProxy proxy
	\item The captured data is stored to a log file
\end{itemize}
	
Small sample of the log output produced by UsaProxy:
\begin{verbatim}
	141.84.8.77 2006-08-27,21:03:00 http://www.kiko.com/ load size=862x389
	141.84.8.77 2006-08-27,21:03:03 http://www.kiko.com/ click coord=249,195 dom=abaaaa
	141.84.8.77 2006-08-27,21:03:24 http://www.kiko.com/ mouseover coord=341,164 dom=abaaaaa img=kf_index.gif
	...
\end{verbatim}
	
\subsection{Interpreting interactions}
To be able to do something with the data logged by UsaProxy I have developed a web application that shows the user three colors and asks them which one they prefer. Based on the events logged by UsaProxy, the web application determines which color the user prefers. 