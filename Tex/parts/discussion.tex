\section{Discussion and Future Work}
\label{sec:discussion}

\subsection{Impact on privacy}
As we have seen in the previous section, using Ajax we are able to capture a lot of possible privacy-infringing data. If we relate the collection of such data to the concept of privacy as described in Section \ref{sec:methodology}, we see that it violates most of the privacy dimensions. 

\subsection{Legal measures}
If we apply the legal framework as described in Section \ref{sec:methodology}, we see that both directives apply in this situation as well. 

\subsection{Technical measures}
One technical measure could be a browser add-on that can recognize the capturing of data using Ajax. Such an addon could inform the user of possible privacy infringements or block data flows from the client device.\\

Another technical measure is to insert noise to obfuscate real user behaviour from the server.

\subsection{Further research}
It would be interesting to see what other possibilities with data collected by Ajax are. An example of a powerful way of interpreting the logged data would be to apply a user interaction model to derive certain properties from users. These properties could then lead to complete user profiles, which can be utilized to recognize visitors based on their earlier visits.