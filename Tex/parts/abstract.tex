% !TeX spellcheck = en_GB
\begin{abstract}
	Ajax is a collection of web technologies utilised to provide a richer user experience when browsing the Internet. It acts as an intermediary layer between the user and the server, which allows for asynchronous communication. This means that the user no longer has to wait for the server to reload entire pages, because data can be returned instantaneously. But there are also downsides to this technology, for example with regard to privacy: Ajax can be used to turn the browser into a surveillance and profiling tool. User activities can be logged more accurately than without Ajax and used to build profiles or draw (presumed) conclusions about users. The objective of this research is to study the relation between the Ajax technology and privacy. First I give an overview of Ajax and its current use. To demonstrate the possibilities of this technology I utilise UsaProxy, a proxy that logs all user interaction on a webpage by means of Ajax. I also developed an experiment that is able to parse these logs and derive properties of a user from it. After discussing the logged data and the results of the experiment I draw conclusions and make predictions about Ajax in terms of technology and privacy. I also discuss legal and technical measures to help prevent Ajax privacy infringements.\\\\
	{\bf Keywords:} Ajax, privacy, web interaction, UsaProxy, profiling
\end{abstract}