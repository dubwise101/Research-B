% !TeX spellcheck = en_GB
\section{Results}
\label{sec:results}
This sections describes the results of the experiment, but first I evaluate the logged data in general.

\subsection{Evaluation of the logged data}
By automatically adding the client-side monitoring UsaProxy script to every visited web page, various information about user behaviour can be identified from the aggregated data of several test sessions.\\

Since the information about all visitors and an ordered listing of all their Web page visits with an accurate time stamp is aggregated in the log file, characteristics such as the paths users take (click through path), the page conversion, and key entry and exit pages can be recognized and classified. Various time-based metrics, such as average time spent on a page or the visit date and time can be retrieved. In that way differences in browsing behaviour and hot spots may be identified to gain insight in which content users prefer most.\\

Analysing the browser window size and scrolling behaviour gives a first insight into the user's preferences. This feature can be easily extended by letting a web application additionally identify the user's browser version, screen resolution, installed plugins, operating system, available bandwidth, and so on.\\

But the real power of the logged interactions lies in the additional event-oriented indicators aggregated in the log file. Besides page entering and window resizing events, clicking, stroking keys, moving the mouse and scrolling give deeper insights into the user's intentions while browsing a website. The time a user spends exploring a Web page before leaving it and his actions performed in the meantime can be accurately identified.\\

The tracked mouse movements visualize the user's focus within a web page, whether they hesitate on other interesting links or text areas before clicking, and recognize content related hot spots.
This may lead to the discovery of where users encounter obstacles such as confusing navigation, difficult to find links, and missing information or where they refer to more dominating (thus distracting) page elements. Furthermore, a non-hesitating mouse movement straight to a link can exhibit a determination in finding particular information on a website. \\

Likewise the scrolling of a web page may signify whether the user is scanning the page or reading a piece of text. Do users scroll or do they evaluate distracting text or image information before they get to the right button? Do users need various attempts to type in data into the right form field?
Did they go through the complete information on a web page before they left? Do they leave the page in-between before they accomplish a task? The answers to these questions may all be drawn from the log file, at least when concerning the aggregated results from large amounts of data.

\subsection{Experiment}
The experiment was conducted in a period of one week, in which a total of 30 people participated. By keeping track of how many interactions were logged to the individual images, the web application derived a top three. This top three was presented to the user, and asked them to confirm correctness. In 26 cases the user indicated that the web application correctly determined their top three images. This demonstrates the fact that user interaction patterns can be used to discover user preferences with high accuracy. This also clears the way for more advanced experiments, in which more interesting metrics (like time) can be calculated from the log and utilised to derive properties of the user. 