% !TeX spellcheck = en_GB
\section{Introduction}
	
%The big picture
Nowadays most web users are familiar with the concept of cookies. They are a web mechanism designed for the functioning of a website. But they are also used for third-party tracking and profiling, which is considered as privacy invasive. That is why the European Union (EU) e-Privacy Directive was amended to require websites to ask their users for consent\footnote{``...the storing of information, or the gaining of access to information already stored, in the terminal equipment of a subscriber or user is only allowed on condition that the subscriber or user concerned has given his or her consent, ...'', \href{http://register.consilium.europa.eu/doc/srv?l=EN\&f=ST\%203674\%202009\%20INIT}{\nolinkurl{http://register.consilium.europa.eu/doc/srv?l=EN\&f=ST\%203674\%202009\%20INIT}} p.76.} when cookies are used for such purposes. But cookies are not the only web technology that can be used to invade a user's privacy: location tracking, web beacons and device fingerprinting can be abused as well. The focus of this research is on Ajax.
	
%History & what is known
Back in the days, browsing the web was a rather static experience: the user triggered HTTP requests by clicking on links of a web page, the server then processed these requests and returned HTML pages and related objects to the client. This meant that the entire page had to be reloaded, while often only some of its content changed. To avoid this, Ajax was used to hide the processing from the user. It is a collection of web technologies, allowing for asynchronous communication, that acts as an intermediary layer between the user and the server. This means that the user no longer has to wait for the server. This makes web pages more interactive, and thus a richer user experience is achieved. The difference between these two web application models is shown in Figure \ref{fig:model}.\\
Some example Ajax applications are the dynamic loading of news articles while scrolling and displaying search results while you type. On the other hand, this results in more fine-grained communication between the client and the server. Since Ajax can be used to collect user activities, it essentially converts the browser into a surveillance and profiling tool. All activities of the user (such as keystrokes and mouse movements) can be logged.\\

\begin{figure}[h]	
	\centering
	\includegraphics[scale=0.39]{images/ajax-fig2.png}
	\caption{The synchronous interaction pattern of a traditional web application (top) compared with the asynchronous pattern of an Ajax application (bottom).}
	\label{fig:model}
\end{figure}
	
%What is unknown
How Ajax can be used to profile website visitors remains unknown. The hypothesis of this research is that data collected using Ajax can be used to build profiles or draw (presumed) conclusions about users which can obviously be privacy invasive. If website visitors can be profiled using Ajax, then a large number of web users could be subject to privacy infringements.\\	
	
%Why it is important
Web applications that use client-side scripting are nearly everywhere: 98 of the 100 most-viewed websites in the United States use client-side JavaScript~\cite{guha2009using}, and half of these use XMLHttpRequest, the asynchronous callback mechanism that characterizes Ajax web applications. Research\footnote{https://gds.blog.gov.uk/2013/10/21/how-many-people-are-missing-out-on-javascript-enhancement/} by the British Government has shown that only 1.1\% of their website visitors are not getting JavaScript enhancements, so the majority of web users are equipped to run Ajax. These numbers combined show that a major part of the Internet could be affected by this research. The fact that there is no extensive research on the relation between Ajax and privacy makes it an urgent topic to study.\\

Another reason for conducting this research is that while there are regulations in place to restrict the collection of data using web technologies, they have not yet been interpreted to apply to Ajax. They have been applied to cookies for example, as they can be used to track website visitors. Websites have to ask the user for explicit permission when such cookies are used. However, using Ajax it is also possible to collect privacy-sensitive data, possibly even more than with cookies, but it runs without a deliberate decision being made by the user. This research aims to draw attention to this situation, so that appropriate action can be taken to address the issue.\\
	
%Summary of approach
The remainder of the paper is organised as follows. I start off by detailing the history of Ajax, its technology and modern day applications in Section \ref{sec:ajax}. I continue by describing the methodology used for this research in Section \ref{sec:methodology}, which is broken down into a definition of privacy, a legal framework, logging user interaction and a description of the experiment. Section \ref{sec:results} studies what implications the logged user behaviour has on privacy. It also describes the results of the experiment. Finally, I discuss the legal and technical measures that can be taken in order to reduce the impact of Ajax privacy invasion in \ref{sec:discussion}, also including what future research can be done from here.\\

I would like to thank my supervisor Gergely Alp\'{a}r for giving me the idea to do this research and guiding me along the way. Colette Cuijpers provided me with some assistance with regard to the legal aspects of this research, for which I would like to thank her as well.\\

The experiment can be found at \url{http://77.249.251.118/demo/images/}.\\ Its source code can be obtained from \url{https://github.com/maartenderks/research-b}.